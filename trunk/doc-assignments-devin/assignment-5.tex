
\documentclass{article}

\usepackage{CJK}
\usepackage{clrscode}

\begin{document}
\begin{CJK}{UTF8}{gbsn}
\title{Algorithm Analysis and Design, Assignment 5}
\author{徐寅斐, 2008212589}
\date{\today}
\maketitle

\section{CLRS, Page 363, Exercises 15.5-4}

\subsection*{思路:}

正如题中所述, $root[i, j - 1] \leq root[i, j] \leq root[i + 1, j]$,在算法中作最内层循环时,
不需要将$r$从$i$到$j$全部遍历一遍,而只需要从$root[i, j - 1]$遍历到$root[i + 1, j]$
即可,即$root[i,r-1]\leq r \leq root[r+1,j]$。
\subsection*{伪码:}

\begin{codebox}
\Procname{$\proc{Optimal-BST}(p,q,n)$}
\li	\For $i \gets 1$ \To $n+1$
\li		\Do $e[i,i-1] \gets q_{i-1}$
\li			$w[i,i-1] \gets q_{i-1}$
		\End
\li	\For $l \gets 1$ \To $n$
\li		\Do \For $i \gets 1$ \To $n-l+1$
\li			\Do $j \gets i+l-1$
\li				$e[i,j] \gets \infty$
\li 			$w[i,j] \gets w[i,j-1]+p_j + q_j$
\li				\For $r \gets root[i, j-1]$ \To $root[i+1,j]$
\li 				\Do $t \gets e[i,r-1] + e[r+1, j] + w[i,j]$
\li						\If $t < e[i,j]$
\li							\Then $e[i,j] \gets t$
\li									$root[i,j \gets r]$
							\End
					\End
			\End
		\End
\li \Return $e$ and $root$
\end{codebox}

\section{CLRS, page 384, Exercises 16.2-2}
\subsection{算法设计描述}
此题可以这样思考:假设小偷已经拿了一包重量不超过$w$的最值钱的东西$P$,并且其中包含
一个物品$i$,那么,$P - i$是除去物品$i$之后的$n-1$件物品中重量不超过$w-w_i$的
最值钱的一包东西,这可以通过cut and paste方法证明。
假设$c[i,w]$表示在前$i$个物品中取出的不超过w重量的最大价值,第$i$个物品为$item_i$,
其重量为$w_i$,价值为$v_i$,并假设前面已经算出了前$i-1$个物品时不超过各个$w$值的最大价值$c[i-1,w]$,
接下来算前$i$个物品中取出不超过$w$重量的$c[i,w]$:如果$item_i$入选,则其余部分是
在前$i-1$个物品中取出的不超过$w-w_i$的最大价值的物品,其价值为$c[i-1,w-w_i]$,
这样总价值为$c[i-1,w-w_i]+v_i$;如果$item_i$没有入选,则表示选出的物品是在前
$i-1$个物品中选出的不超过$w$重量的最大价值的物品,其价值为$c[i-1,w]$,我们需要做的,
就是判断前面给出的两个价值那个更大。递归表达式:\newline

\begin{displaymath}
c[i,w] = \left \{ \begin{array}{ll}
                          0 	&i=0 \ or \ w = 0\\
                          c[i-1,w] 		& i > 0 \ and \ w_i > w\\
                          max(c[i-1,w-w_i]+v_i, c[i-1,w]) 	& i > 0 \ and \ w_i \leq w
                  \end{array} \right.
\end{displaymath}

所以,对于$c$数组的计算,我们可以从$i=0$开始逐行计算,$i=0$时全部$c[0,w]$初始化成0,
计算到最后$c[n,w]$中便是存储的最大的价值,此题不需要额外的数组去记录最后被选择的物品,
直接用$c$数组就行:从i=n,w = W开始,当$c[i,w] \neq c[i-1, w - w_i]$时,
说明$item_i$被选中,$i = i-1, w = w-w_i$,继续寻找下一个;否则没被选中,则
$i = i-1, w = w$,继续寻找下一个。

\section{CLRS, page 384, Problems 16.2-6}

Problem 9.2 中已经给出了在线性时间内找出带权中间值,在这个问题中,我们可以把每个物品
的价重比($v/w$)看成物品的权重,因此我们可以在线性时间中找出$v/w$的中间值m,接下来再用
线性的时间将所有物品分成3组:$G = \{ i | v_i/w_i > m\}$,$E = \{ i | v_i/w_i = m\}$,
$L = \{ i | v_i/w_i < m\}$,很显然,先从价重比最高的开始拿,即从G中开始拿,
再拿之前,先计算出$G$、$E$的总质量:$W_G = \sum_{i \in G}w_i$,$W_E = \sum_{i \in
E}w_i$,这个也需要线性时间完成。
接下来分情况讨论:
\begin{itemize}
\item 如果 $W_G > W$,则继续递归分解G,从中找出质量为$W$的物品。
\item 如果 $W_G <= W$ 并且$W_G + W_E \geq
W$,则说明用G和E中的物品可以填满整个包裹,那么就将G中的物品全部取出,
并用E中的物品将包括填满(优于E中的物品价重比是一样的,所以不需要考虑拿哪个)。
\item 如果 $W_G + W_E < W$,说明G和E中的物品还不能填满包裹,则取出G和E中的所有物品,并递归的
调用上述分解过程,从L中取出质量为$W - W_G - W_E$的物品。
\end{itemize}
算法复杂度分析:此算法中使用了递归,并且每次分解时自问题最大不超过原问题的二分之一,并且只有一个子问题,
每次递归使用的时间都是线性的。因此,递归表达式可写为$T(n) \leq T(\frac{n}{2})+\Theta(n)$,因此,
根据祖定律,时间复杂度为$O(n)$。

\section{CLRS, page 392, Problems 16.3-6}
\subsection{算法描述}
可以使用一颗满三叉数来表示三叶子分别表示0、1、2,算法于霍夫曼编码基本类似,使用一个最小队列,先把所有的节点压入队列中,
然后每次新建一个节点,取出频率最小的三个节点作为新建节点的叶子,然后把新节点插入队列中,其频率为
三个子节点的频率之和。如此循环,直到最后三个节点被取出挂到新建的根节点,由于初始节点如果为偶数时,
最后会剩两个节点,而不是三个,因此在初始节点为偶数的情况下,先增加一个频率为0的虚拟节点。
\subsection{证明}
\textbf{Lemma 1:} Let $C$ be an alphabet in which each character $c \in C$ has
frequency $f[c]$. Let $x,y$ and $z$ be three characters in $C$ having the lowest
frequencies. Then there exists an optimal prefix code for $C$ in which the code
words for $x,y$ and $z$ have the same length and differ only in the last
digit.\newline

证明:设$a,b,c$为最优树$T$中具有最大深度的兄弟节点,不失一般性,我们假设$f[a]\leq f[b] \leq f[c]$
且$f[x]\leq f[y] \leq f[z]$。由于$x,y,z$是频率最低的三个点,所以$f[x]\leq f[a]$,$f[y]\leq
f[b]$,$f[z]\leq f[c]$。交换$a,x$的位置形成新树$T'$,则两棵树之间的代价差为:\newline
\begin{eqnarray*}
B(T) - B(T') &=& \sum_{c\in C}f(c)d_T(c) - \sum_{c\in C}f(c)d_{T'}(c)\\
 &=& f[x]d_T(x)+f[a]d_T(a) - f[x]d_{T'}(x)+f[a]d_{T'}(a)\\
 &=& f[x]d_T(x)+f[a]d_T(a) - f[x]d_T(a)+f[a]d_T(x)\\
 &=& (f[a]-f[x])(d_T(a)-d_T(x)) \geq 0
\end{eqnarray*}

所以,$x$和$a$交换不会增加代价。同理,交换$y$和$b$,$z$和$c$同样也不会增加代价。因此,在将所有
三对节点交换之后,所得的树仍然是一颗最优树,而且$x,y,z$是三个同样具有最大深度的兄弟节点,从而证明了
Lemma 1。

所以,构造一颗最优树的过程可以从合并三个频率最低的字符开始,这就是我们的贪心选择,我们可以认为一次合并
的代价是三个合并节点的频率之和。在每一次合并中,本算法都选择一个代价最小的合并。\newline

\textbf{Lemma 2:} Let $C$ be a given alphabet with frequency $f[c]$ defined for
each character $c \in C$. Let $x,y,z$ be three characters in $C$ with minimum
frequency. Let $C'$ be the alphabet $C$ with characters $x, y, z$ removed and
(new) character $w$ added, so that $C' = C - {x, y, z} \cup {w}$; define $f$ for
$C'$ as for $C$, except that $f[w] = f[x] + f[y] + f[z]$. Let $T'$ be any tree
representing an optimal prefix code for the alphabet $C'$. Then the tree $T$,
obtained from $T'$ by replacing the leaf node for $w$ with an internal node
having $x, y, z$ as children, represents an optimal prefix code for the alphabet
$C$.\newline

证明:优于这两棵树的大部分结构相同,我们可以考虑用$T'$的代价$B(T')$来表示$T$的代价$B(T)$。对于
每一个$c \in C - \{x,y,z\}$,都有$d_T(c) = d_{T'}(c)$,因此$f[c]d_T(c)=f[c]d_{T'}(c)$ 。
又因为$d_T(x,y,z) = d_{T'}(w)=1$,所以

\begin{eqnarray*}
f[x]d_T(x) + f[y]d_T(y) + f[z]d_T(z) &=& (f[x]+f[y]+f[z])(d_{T'}(w)+1)\\
									 &=& f[w]d_{T'}(w) + (f[x]+f[y]+f[z])
\end{eqnarray*}

因此我可以得出:\newline
$B(T) = B(T') + f[x]+f[y]+f[z]$\newline
即\newline
$B(T') = B(T) - f[x]-f[y]-f[z]$

接下来用反证法来证明,假设$T$不是$C$的最优前缀编码,那么存在一颗树$T''$,使得$B(T'')<B(T)$。
不失一般性(Lemma 1),$T''$有$x,y,z$三个兄弟节点。设$T'''$是将$T''$中的$x,y,z$替换为$w$得到
的树,其中频率$f[w] = f[x]+f[y]+f[z]$。则

\begin{displaymath}
\begin{array}{ccl}
B(T''') & = & B(T'') - f[x]-f[y]-f[z]\\
		& < & B(T) - f[x]-f[y]-f[z]\\
		& = & B(T') 
\end{array}
\end{displaymath}

这与我们的假设的($T'$为$C'$上的任意一颗最有前缀编码树)矛盾,因此,假设不成立,所以$T$
是字母表$C$上的最优编码。

综上,此算法产生了最优前缀编码。
\end{CJK}
\end{document}
