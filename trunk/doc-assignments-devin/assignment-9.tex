\documentclass{article}

\usepackage{CJK}
\usepackage{clrscode}

\begin{document}
\begin{CJK}{UTF8}{gbsn}
\title{Algorithm Analysis and Design, Assignment 7}
\author{徐寅斐, 2008212589}
\date{\today}

\maketitle

\section{CLRS, Page 931, Problem 32.4-6}

\subsection*{思路}

这里的复杂度有$O(m^3|\Sigma|)$,主要是因为在计算状态迁移函数时使用了大量
的Naive的字符串匹配,其实不用这么繁琐。当前k-1个状态前移已经计算出来后,从k状
态开始的状态前移只需要根据几个字符以及前面的状态推算出来,不需要进行大量的字符
串匹配,这里借用了KMP算法中前缀子串的概念,我们可以很容易得证明:如果$q <m$且
$P[q+1] = a$,则$\delta(q,a) = q+1$;如果$q = m$或者 $P[q+1]\neq a$,
则$\delta(q,a) = \delta(\pi[q],a)$。\newline\newline

证明:
\subsubsection*{a.如果$q <m$且$P[q+1] = a$,则$\delta(q,a) = q+1$}
因为$P[q+1] = a$,所以$P_qa = P_qP[q+1] = P_{q+1}$,因此$P_{q+1} \sqsupset
P_qa$,所以$\sigma(P_qa) = q+1$,因此$\delta(q,a) = q+1$。

\subsubsection*{b.如果$q = m$或者 $P[q+1]\neq a$,则$\delta(q,a) = \delta(\pi[q],a)$}
\begin{itemize}
  \item $q<m$且$P[q+1]\neq a$时,这时考虑$\pi[q]$,由于$\pi[q] = max{k: k<q,P_k
  \sqsupset P_q}$,即最大的$k$,使得$P[1]\ldots P[k] = P[q-k+1]\ldots
  P[q]$。这时接受$a$之后转移的状态必然小于q+1,而当前已接受的字符序列中,最后$k$位为小于q的P的最大前缀,
  即现在的状态可以看成是在k状态时接受字符$a$,因此,可以得到$\delta(q,a) = \delta(\pi[q],a)$。
  \item $q=m$时,由于已经没有状态$q+1$,只能从当前序列中找出小于q的P的最长前缀,在此基础上进行
  状态迁移,而这个最长前缀必然为$\pi[k]$,因此$\delta(q,a) = \delta(\pi[q],a)$。
\end{itemize}

\subsection*{伪码}

计算前缀函数$\pi$:

\begin{codebox}
\Procname{$\proc{Compute-Prefix-Function}(P)$}
\li $m \gets length[P]$
\li $\pi[1] \gets 0$
\li $k \gets 0$
\li \For $q \gets 2$ to $m$
\li		\Do \While $k > 0 $ and $P[k+1] \neq P[q]$
\li				\Do $k \gets \pi[k]$
				\End
\li			\If $P[k+1] = P[q]$
\li				\Then $k \gets k+1$
				\End
\li			$\pi[q] \gets k$
		\End
\li \Return $\pi$		
\end{codebox}

计算状态迁移函数$\delta$:

\begin{codebox}
\Procname{$\proc{Compute-Transition-Function}(P,\Sigma)$}
\li $\pi \gets \proc{Compute-Prefix-Function}(P)$
\li $m \gets length[P]$
\li \For each character $a \in \Sigma$
\li 	\Do \If $P[1] = a$
\li				\Then $\delta(0,a) \gets 1$
\li				\Else $\delta(0,a) \gets 0$
				\End
		\End
\li \For $q \gets 1$ to $m$
\li		\Do \For each character $a \in \Sigma$
\li				\Do \If $q = m$ or $P[q+1] \neq a$
\li						\Then $\delta(q,a) \gets \delta(\pi[q],a)$
\li						\Else $\delta(q,a) \gets q+1$
						\End
				\End
		\End
\li \Return $\delta$
\end{codebox}

\subsection*{复杂度分析}

我们已经知道,计算前缀函数的复杂度是$O(m)$的,而计算状态迁移函数时,很容易看出,
3 - 6行一重循环,复杂度为$O(m)$, 7 - 11行为嵌套循环,外层循环$m$次,内层循环
$|\Sigma|$次,因此复杂度为$O(m|\Sigma|)$,所以,总复杂度为$O(m|\Sigma|)$。


\end{CJK}
\end{document}
