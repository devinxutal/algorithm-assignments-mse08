\documentclass{article}

\usepackage{CJK}
\usepackage{clrscode}

\begin{document}
\begin{CJK}{UTF8}{gbsn}
\title{Algorithm Analysis and Design, Assignment 7}
\author{徐寅斐, 2008212589}
\date{\today}
\maketitle

\section{CLRS, Page 549, Problem 22.3-12}

\subsection*{思路}
从图中的每个节点开始进行一次DFS搜索,如果没有发现任何前边和
存在于树内的交叉边,则说明这个有向图是单连通的。 准确的说,``从每个节点开始进
行一次DFS搜索,没有发现前边和交叉边''是``这个有向图
是单连通的''的充要条件。\newline

证明:

[充分性]如果一个有向图是单连通的,可以证明无论从哪个节点开始进行DFS,都不会存在前边,
假设存在一条前边(u,v),则肯定存在另一条从u到v的路径,已经被DFS搜索时标注过了,这样的话就存在了
两条u到v的路径,这与``单连通''的定义冲突,因此,不可能存在前边。

同样的,假设存在一条树内的交叉边(u,v),由于u和v在同一棵树中,则肯定拥有同一个根节点r,则从r必然
存在两条到达v的路径,一条经过u而另一条不经过u。这也与单连通性冲突,因此不可能存在树内的交叉边。

[必要性]在没有找到任何前边和树内的交叉边时,假设这个图不是单连通的,则必然存在
一些点对,从某个点出发存在至少两条路径到达另一个点,我们从中选择一个点对u和v,
使得u到v拥有这些点对中最短的路径(假设为$p_s$),本算法从u开始对图进行DFS搜索时,肯定会产生
一条u到v的路径p.由于这两个点还有另外一条路径,假设为p',如果p'的长度为1,则说明p'
就是直接从u连到v的一条边,则这条边(u,v)就是一条前边。如果p'的长度不是1,假设p'上离v最近的
点为x,则(x,v)就是一条交叉边。这与我们的条件冲突。假设不成立。
因此,``从每个节点开始进行一次DFS搜索,没有发现前边和树内的交叉边''是``这个有向图是单连通的''的必要条件。

综上所述,``从每个节点开始进行一次DFS搜索,没有发现前边和树内的交叉边''是``这个有向图是单连通的''的充要条件。\newline

复杂度分析: 本算法对每个节点进行了一次DFS搜索,每次的时间复杂度为$O(V+E)$,所以总的时间复杂度为$O(V(V+E))$。

\subsection*{伪码}
\begin{codebox}
\Procname{$\proc{Singly-Connected}(G)$}
\li \For each vertex $u \in V[G]$
\li 	\Do \If $\proc{DFS}(G, u)$ is not true
\li				\Then \Return false;
				\End
		\End
\li \Return true
\end{codebox}

\begin{codebox}
\Procname{$\proc{DFS}(G,u)$}
\li \For each vertex $v \in V[G]$
\li 	\Do $color[v] \gets WHITE$
\li 		$\pi[u] \gets NIL$
		\End
\li $time \gets 0$
\li \Comment 每次DFS只需要搜索以u开始的第一棵树即可
\li \Return $\proc{DFS-Visit}(u)$
\end{codebox}

\begin{codebox}
\Procname{$\proc{DFS-Visit}(u)$}
\li $color[u] \gets GRAY$
\li $time \gets time + 1$
\li $d[u] \gets time$
\li \For each $v \in Adj[u]$
\li 	\Do 
\li			\If $color[v] = BLACK$
\li				\Then \Return false
				\End
\li			\If $color[v] = WHITE$
\li 			\Then 	$\pi[v] \gets u$
\li 					\If $\proc{DFS-Visit}(v)$ is not true
\li	 						\Then \Return false
							\End
				\End
		\End
\li $color[u] \gets BLACK$
\li $f[u] \gets time \gets time + 1$
\li \Return true;
\end{codebox}

\end{CJK}
\end{document}
