\documentclass{article}

\usepackage{CJK}
\usepackage{clrscode}

\begin{document}
\begin{CJK}{UTF8}{gbsn}
\title{Algorithm Analysis and Design, Assignment 7}
\author{徐寅斐, 2008212589}
\date{\today}

\maketitle

\section{CLRS, Page 549, Problem 22.3-12}

\subsection*{a}
假设在新增的一条边为$(u,v)$,这表示,所有能够到达$u$的节点能够到达$v$,因此所有能够到达$u$的节点
能够到达$v$能够到达的所有节点。因此,对于任意两点$i,j$,如果边$(i,u)$和$(v,j)$都存在,则
$(i,j)$也存在。

假设算法建立在一个$|V| \times |V| $矩阵的矩阵上,初始状态下不存在边,因此矩阵中都为0,但
对角线上都为1,因为所有的节点都可以到达自己。则算法如下:

\begin{codebox}
\Procname{$\proc{Update-Transitive-Closure}(u,v)$}
\li \For $i \gets 1$ to $|V|$
\li 	\Do \For $j \gets 1$ to $|V|$
\li 			\Do \If $T[i, u] = 1$ and $T[v, j] = 1$
\li						\Then $T[i, j] \gets 1$
						\End
				\End
		\End
\end{codebox}

本算法对矩阵做了一次遍历,因此复杂度为$\Theta(V^2)$。

\subsection*{b}
假设一个含有$n$个顶点的图中只存在这样一条路径$v_1 \to v_2 \to \ldots \to v_n$,那么,这个
图的传递闭包中含有$n(n+1)/2$条边(包括所有对角线上的1)。在计入一条边$(v_n, v_1)$之后,任意两个
顶点都可以互相到达,因此,在其传递闭包中有$n^2$条边,也就是说,有$n^2 - n(n+1)/2 = \Theta(V^2)$
条边被更新了,因此,更新操作至少需要花费$\Omega(n^2)$的时间。

\subsection*{c}
分析a中的算法可以看出,内层循环中存在了很多重复劳动。算法中我们将所有的$(i,u),(v,j)$都存在的点对$(i,j)$
之间的边加上,即令$T[i,j]=1$,但是,如果$T[i,v]$已经为1,则对于所有$T[v,j]=1$的点$j$来说,$T[i,j]$
早就已经是1了,不需要重复设置。因此,这里的改进算法是不去重复设置这些已经为1的地方:

\begin{codebox}
\Procname{$\proc{Update-Transitive-Closure}(u,v)$}
\li \For $i \gets 1$ to $|V|$
\li 	\Do \If $T[i,u = 1]$ and $T[i,v] = 0$
\li 			\Then \For $j \gets 1$ to $|V|$
\li						\Do \If $T[v,j] = 1$
\li 							\Then $T[i,j] = 1$
								\End
						\End
				\End
		\End
\end{codebox}

这个算法虽然有两层循环,在插入$V^2$条边的时候看似需要$O(V^4)$的时间,但其实不需要:
\begin{itemize}
  \item 由于可能有$O(V^2)$条边插入,因此算法前两行会被执行$O(V^3)$次
  \item
  第三行到第五行,由于第二行有一个条件判断,只有当$T[i,v]=0$时才会执行,又因每次运行完这三行后至少减少一个0的位置($T[i,v]$),矩阵中总共只有$V^2$个位置,因此,最后三行最多只会被执行$V^2$次,每次执行时间为$O(V)$,总时间花费也是$O(V^3)$。
  \item 因此,总的时间复杂度为$O(V^3)$
\end{itemize}
\end{CJK}
\end{document}
