%%This is a very basic article template.
%%There is just one section and two subsections.
\documentclass{article}

\usepackage{CJK}

\begin{document}
\begin{CJK}{UTF8}{gbsn}
\title{Algorithm Analysis and Design, Assignment 3}
\author{徐寅斐, 2008212589}
\date{\today}
\maketitle

\section{CLRS, page 161, Problems 7.3}
\subsection*{(a)}

当$j-i<3$时,该排序算法对这几个数进行了两两排序,结果肯定有序。

假设当$j-i<n$时可以对数组进行排序,则当$j-i=n$时,设$k =
\lfloor
\frac{n+1}{3}\rfloor$,这时的k是小于等于总个数的三分之一的。

在第一次递归排序时算法对$[i,j-k]$进行排序,排序完成后$[i+k,j-k]$存储的是前面排序范围
内最大的几个数,可以保证,在前面$k$个数不可能是最大的$k$个数。

第二轮对[i+k,j]内的数进行排序,由于数组的前k个数不可能含有最大的k个数,这样,数组中最大的$k$个数肯定已经被排序好放在[j-k+1,j]的位置了。

第三次排序对[i,j-k]排序,由于最大的k个数已经就位,这次排序过后,所有的数都已经排好序了。

假设成立,这个明完毕。

\subsection*{(b)}

由于这个算法的递归深度不会变化,所以不存在什么最坏情况(无非是多交换几次,对复杂度数量级没有影响)。根据排序算法的过程,可以看出排序算法的递推式为$T(n) =
3T(\frac{2n}{3}) +1 $,根据组定律,$T(n) = \Theta( n^{{\log_b}a}) =
\Theta(n^{{\log_\frac{3}{2}}3}) \approx \Theta(n^{2.7})$。

\subsection*{(c)}
其他排序算法的最坏情况最多是$O(n^2)$,比这个算法的最坏情况好。

\section{CLRS, page 173, Exercises 8.3-4}
\subsection{Show how to sort n integers in the range $0$ to $n^2 - 1$ in O(n)
time.} 
答:

可以将这些数看成n进制的数,即看成
\newline

$k = \alpha n+\beta$,其中$0 \leq \alpha,\beta < n$
\newline

这样,这些数就可以看成radix为n的两位数,可以用两次countingsort来实现排序,每次countingsort花费的时间为$\Theta(n+n)
= \Theta(n)$,所以排序的总复杂度也为$\Theta(n)$。
\end{CJK}
\end{document}
