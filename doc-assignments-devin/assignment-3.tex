%%This is a very basic article template.
%%There is just one section and two subsections.
\documentclass{article}

\usepackage{CJK}

\begin{document}
\begin{CJK}{UTF8}{gbsn}
\title{Algorithm Analysis and Design, Assignment 3}
\author{徐寅斐, 2008212589}
\date{\today}
\maketitle

\section{CLRS, page 161, Problems 7.3}
\subsection{Why do we analyze the average-case performance
of a randomized algorithm and not its worst-case performance?}
答:

对于一个算法的随机化版本,它并不会改善最坏情况下的性能,它的目的是保证它不会在特定的情况下遇到最坏
情况,这样就使得这个算法不用依赖元素的顺序,这使得算法在特定情况下遇到最坏情况的概率小了很多。

\subsection{During the running of the procedure RANDOMIZED-QUICKSORT, how many calls are
made to the random-number generator RANDOM in the worst case? How about in the best
case? Give your answer in terms of Θ-notation.}
答:

暂时没想

\section{CLRS, page 173, Exercises 8.3-4}
\subsection{Show how to sort n integers in the range $0$ to $n^2 - 1$ in O(n)
time.} 
答:

可以将这些数看成n进制的数,即看成
\newline

$k = \alpha n+\beta$,其中$0 \leq \alpha,\beta < n$
\newline

这样,这些数就可以看成radix为n的两位数,可以用两次countingsort来实现排序,每次countingsort花费的时间为$\Theta(n+n)
= \Theta(n)$,所以排序的总复杂度也为$\Theta(n)$。
\end{CJK}
\end{document}
